  %%GLOSARIO
  \newdefinition{api}{API}{APIs}{La interfaz de programación de aplicaciones es un conjunto de subrutinas, funciones y procedimientos que ofrece cierta biblioteca para ser utilizado por otro software como una capa de abstracción}
  \newdefinition{framework}{framework}{frameworks}{Estructura conceptual y tecnológica de asistencia definida que sirve de base para la organización y desarrollo de software}
  \newacronym{bbdd}{BBDD}{Bases de Datos}
  \newacronym{bd}{BD}{Base de Datos}
  \newacronym{bdr}{BDR}{Bases de Datos Relacionales}
  \newacronym{bdo}{BDOG}{Bases de Datos Orientadas a Grafos}
  %%%%%%%%%%%%%%%%%%%%%%%%%%%%%%%%%%%%%%%%%%%%
  
\chapter{Estado del Arte\label{CAP:ESTADOARTE}}
  En la actualidad, existen muchas herramientas de desarrollo que facilitan la implementación de esta aplicación. A continuación describiremos los componentes esenciales que son relevantes para el diseño y sus variantes más útiles.
  
  \section{Bases de Datos}
    Se pueden enumerar numerosos tipos de \ac{bbdd} dependiendo de la estructura que utilizan para almacenar sus datos. Los dos principales a considerar para esta aplicación son las \ac{bdr} y las \ac{bdo}.
    
    Las \ac{bdr} siguen el modelo relacional, que consiste en guardar los datos en tablas y relaciones. Mientras que las \ac{bdo} guardan los datos en nodos (también llamados vértices) relacionados con otros nodos mediante aristas (también llamadas relaciones). En este apartado se analizará principalmente a PostgreSQL\cite{postgres}.
    
    Las \ac{bdo} llevan mucho tiempo inventadas, pero fueron eclipsadas en su momento por las \ac{bdr}, las cuáles son más sencillas y por tanto más populares. Por ese mismo motivo, actualmente existen escasas herramientas de \ac{bdo}. En este apartado se analizará principalmente a Neo4J\cite{neo4j}.
    
    PostgreSQL es una \ac{bdr} que fue creada en 1989 e implementada en C. Entre sus características más útiles se destaca su alta concurrencia, es decir, permite que mientras un proceso modifica una tabla, otros accedan a la misma sin necesidad de bloqueos, y así permitiendo a cada usuario obtener una visión consistente. Además permite la ejecución de funciones, procedimientos almacenados y triggers (también llamados disparadores).
    Por estas razones, y por su longevidad en el mercado, es una herramienta más clásica con muchas funcionalidades y que tiene muchos más años de soporte, lo que la hace más robusta y versátil.
  
    Por otra parte, Neo4J es una \ac{bdo} que fue creada en 2007 usando Java y se basa en el uso de "grafos de propiedad". Estos son un tipo de grafos dirigidos, con peso en las relaciones entre nodos, y que poseen etiquetas y propiedades en los nodos y relaciones entre ellos. Neo4J tiene mejor rendimiento que las \ac{bbdd} relacionales y las no relacionales. Esto es porque, aunque las consultas de datos aumenten exponencialmente, el rendimiento de Neo4j no desciende (frente a lo que sí sucede con las BD relacionales). Esto es debido a que las \ac{bdo} responden a las consultas actualizando el nodo y las relaciones de esa búsqueda y no todo el grafo completo y eso optimiza mucho el proceso.
    Además, cabe destacar su flexibilidad y escalabilidad, pues cuando aumentan las necesidades, las posibilidades de añadir más nodos y relaciones a un grafo ya existente son enormes.
    En general, por su naturaleza estas \ac{bbdd} destacan en eficiencia si lo que se busca es optimizar las búsquedas basadas en relaciones entre nodos. 
    
    %distancias
    %//modulo geolocalizacion
    
  \section{Framework}
    A la hora de considerar el desarrollo de esta aplicación, se ha decidido usar un \dfn{framework}. De esta manera se facilita su creación, pues le dotamos de un esqueleto estandarizado en el que es más fácil colaborar.
    
    Los dos \dfnpl{framework} más populares del momento para aplicaciones web desarrolladas en python son Django\cite{django} y Flask\cite{flask}.
    
    Ambos proporcionan una estructura a la aplicación, siendo la de Django más restrictiva y estandarizada. A pesar de ello, Django proporciona muchas herramientas de desarrollo integradas.
    
    Mientras tanto, Flask se caracteriza por su flexibilidad y simpleza. Proporciona menos herramientas de desarrollo, pero se puede solucionar fácilmente mediante la inclusión de plugins.
    
  
  \section{API de mapas}
  
    Existen muchas \dfnpl{api} disponibles para implementar mapas en javascript, la más usada y completa en el mercado actualmente es la de Google Maps\cite{gmaps}. Al ser tan utilizada, se ha convertido prácticamente en el estándar para mapas en la web. Y por tanto, contiene muchas funcionalidades, mapas muy actualizados, una documentación muy cuidada y un buen soporte por parte de los desarrolladores de la \dfn{api}.
    
    El problema con esta \dfn{api} es que fue gratuita hasta el 11 de junio de 2018, pero desde entonces se ha vuelto de pago, y esto conlleva un gasto mensual dependiendo del número de transacciones con la \dfn{api}. Actualmente permite el uso de la \dfn{api} con un coste máximo de 200\$ mensuales de forma gratuita. Además, tiene la desventaja de que no permite su uso offline, por lo que una conexión a internet es necesaria.

    
    Por tanto, se han investigado las alternativas opensource y gratuitas existentes. Las dos \dfnpl{api} principales en este sector son OpenLayers\cite{openlayers} y Leaflet\cite{leaflet}.
    
    OpenLayers es la alternativa más clásica de implementación de mapas opensource. Posee una gran cantidad de funcionalidades integradas como, por ejemplo, visualización de mapas 3D y permitir mostrar rápidamente grandes conjuntos de datos vectoriales.
    
    Leaflet es una librería javascript más moderna que OpenLayers. Por tanto, posee una mejor arquitectura y diseño interno. La idea tras Leaflet es la de una librería más simple y sencilla de utilizar. Por tanto, posee una documentación de la API más cuidada y actualizada. No tiene tantas funcionalidades integradas como Openlayers, pero eso lo suple con la gran variedad de plugins que se pueden añadir.
    
    A diferencia de Google Maps, estas alternativas, no poseen sus propios mapas, por tanto, tienen que hacer uso de un mapa de terceros. En este caso, el proveedor de mapas gratuitos por excelencia es OpenStreetMap\cite{osm}.
    %leaflet es dificil hacer lo de las carreteras
    %%Consulta BBDD osm
  
  \section{Librerías Python}
    Usamos SQLAlchemy
    Usamos Pandas
  %Esto lo voy a mover a diseño  