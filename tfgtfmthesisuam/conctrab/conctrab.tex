\chapter{Conclusiones y trabajo futuro\label{CAP:CONCTRAB}}
  \section{Conclusiones\label{SEC:CONCLUSIONES}}
    TODO
  \section{Trabajo futuro\label{SEC:TRABAJO}}
    A continuación se detallarán las distintas consideraciones que se tendrán en cuenta de cara a mejorar el trabajo ya hecho y ampliar su funcionalidad y alcance:
    
    \begin{itemize}
      \item Reconocimiento automático de formatos de ficheros de datos.
      Actualmente para definir los formatos de los ficheros de datos de antenas y llamadas telefónicas se realiza mediante un fichero JSON. 
      Sería interesante realizar un algoritmo que identificara automáticamente columnas de los ficheros relevantes en base a la cabecera de las columnas o al formato de los datos.
      
      \item Filtrar teléfonos en base a las carreteras cercanas. 
      La API de Maps tiene funcionalidades que permiten identificar carreteras cercanas (Véase Snap to Roads \cite{snaproad}). De esta manera se podrían filtrar llamadas que se realicen a alta velocidad siguiendo una carretera.
      
      \item Triangulación de trayectorias de terminales.
      Con los datos simulados actuales, sólo se puede estimar la ubicación de un terminal por áreas alrededor de las antenas a las que se conecta.
      Pero con un muestreo de datos reales suficiente sería posible estimar la ubicación de un terminal de forma más exacta mirando sus instantes de tiempo y las intersecciones entre los rangos de las distintas antenas. 
    \end{itemize}