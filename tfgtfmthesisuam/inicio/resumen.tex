Las fuerzas y cuerpos de seguridad muchas veces se enfrentan a la tarea de encontrar sospechosos para un crimen ya cometido, o, en el caso de una desaparición, de encontrar a la víctima. Para resolver estos crímenes, muchas veces sólo disponen de una ubicación aproximada del crimen realizado, o del teléfono de la víctima. Por tanto, accediendo a un registro de llamadas y antenas telefónicas, deberían ser capaces de facilitar el proceso de resolución del crimen.

Aquí entra este trabajo, pues consiste en desarrollar una aplicación que provea de información a un analista para poder identificar a los posibles sospechosos de un crimen conociendo la ubicación y el intervalo temporal de realización de dicho crimen. Además de permitir seguir la ubicación de determinados teléfonos y las trayectorias que siguen de cara a poder establecer sus posibles cómplices en el crimen. 

Para ello, se ha desarrollado un módulo de inserción en base de datos de ficheros con un formato configurable. Y se ha desarrollado una interfaz web donde se pueden realizar, mediante formularios, las consultas en bases de datos. Los resultados se muestran mediante un mapa interactivo en Google Maps, donde se pueden visualizar los datos en detalle al clicar en las distintas ubicaciones del mapa.


\palabrasclave{aplicación, base de datos, Google Maps, llamadas telefónicas, crimen, sospechosos}