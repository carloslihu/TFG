\section{Dificultades Encontradas\label{SEC:DIFICULTAD}}
  \subsection{API de mapas}
    Durante la implementación del módulo de mapas, se optó en un principio por tratar de usar librerías y maptiles opensource. Más concretamente, usamos la librería de Leaflet junto a los maptiles proporcionados por OpenStreetMap.
    El problema surgió cuando comprobamos que con esta API contenía una funcionalidad muy rudimentaria y poco usada para la búsqueda de carreteras cercanas a un punto en el mapa. Esta funcionalidad está provista por la API de overpass de OpenStreetMap. Esta API resulta ser rudimentaria 
    Por ese motivo, se tuvo que buscar una alternativa que tuviera esta funcionalidad. Esa alternativa acabó siendo la API de Google Maps, que como explicamos, es de pago, pero es la más completa y usada del mercado.
    Tiene sus desventajas, pues no permite descargarse los mapas offline, y tiene que mantenerse conectado a la red para cargar los maptiles.
    %overpass
   \subsection{Simulado de datos}
    Para empezar, se ha comprobado que los datos que poseen las distintas compañías sobre antenas y teléfonos tienen formatos completamente distintos, siendo muy difícil unificar la lectura de ficheros en una misma tabla de la BD.
    Además, debido a la sensibilidad de los datos que se manejan, no hemos podido probar la aplicación con datos reales captados de repetidores de antenas verdaderos. Por ese motivo, ha resultado muy difícil simular casos realistas para comprobar que la aplicación funciona correctamente. Al final se han simulado datos con unos formatos que se han considerado básicos y que todos los ficheros deberían tener como mínimo.
    %datos han sido dificiles de simular
    