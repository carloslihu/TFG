\section{Diseño\label{SEC:DISENO}}
  Tras la evaluación de las diferentes herramientas en el estado del arte, aquí asignamos las que usaremos finalmente, junto al criterio que hemos seguido a la hora de hacerlo.
  
  \subsection{Base de Datos}
    La BD que usaremos será PostgreSQL. Esto es debido a que para esta aplicación, se ha observado que no existe mucha complejidad a la hora de almacenar los datos.
    
    Para crear los formatos de los datos de la BD hemos tenido de referencia datos reales proporcionados, y hemos simulado un modelo básico que contiene todos los campos que hemos visto comunes y necesarios.
    %Antenas y llamadas?
    Los formatos de los datos de antenas los hemos sacado de OpenCellid \cite{opencellid}
    %detallar campos
    Para definir los campos de las llamadas, se ha accedido a datos de compañías telefónicas privadas, y se han comparado para encontrar los campos comunes y necesarios.
    %detallar campos
  \subsection{Framework}
    Debido a que la envergadura del proyecto no es muy grande, el framework que se utilizará será Flask, pues es más sencillo que Django, y proporciona prácticamente las mismas herramientas. Por lo que en este caso primaremos la simpleza, para que el proyecto no sea innecesariamente complicado.

  \section{API de mapas}
    Finalmente se ha decidido usar la \dfn{api} de Google Maps\cite{gmaps} a pesar de que su uso conlleve un gasto mensual. Esto es porque, tras probar las diferentes alternativas, se ha comprobado que esta \dfn{api} es la única que tiene un módulo de Roads para detectar carreteras cercanas que funciona aceptablemente y que tiene un soporte y una documentación más cuidada.
  %Diagrama casos uso
  %Diagramas para explicar codigo