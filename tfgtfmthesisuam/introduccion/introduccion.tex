\chapter{Introducción\label{CAP:INTRODUCCION}}
  
  \section{Motivación\label{SEC:MOTIVACION}}
    A diario, se cometen crímenes en ubicaciones recónditas y pasan desapercibidos. Una vez que se descubren tales crímenes, ya es muy tarde para ubicar a los sospechosos, o en el caso de una desaparición o secuestro, a las víctimas de tales crímenes.
    
    Para resolver dichos crímenes, las fuerzas y cuerpos de seguridad muchas veces disponen únicamente de una ubicación aproximada en donde se ha cometido el crimen para identificar a los posibles sospechosos. En otros casos, en los que interesa más localizar a una víctima, sólo disponen de su teléfono para comenzar la investigación de su paradero.
    
    Por tanto, a petición de la Guardia Civil, se decide implementar una herramienta con los requisitos establecidos por ellos.
    La funcionalidad de esta herramienta consiste en, a partir de los datos provistos por distintas compañías de operadoras telefónicas, ser capaz de proveer a un analista de información y un cierto grado de reconocimiento automático para poder resolver estos casos.
  
  
  \section{Objetivos\label{SEC:OBJETIVOS}}
    A continuación listaremos los objetivos que debe cumplir la aplicación para ser satisfactoria:
    \begin{objetive}
      \item Permitir definir los formatos de los ficheros de datos a insertar en la Base de Datos.
      \item Permitir insertar los datos en la Base de Datos usando dichos formatos.
      
      \item Permitir conocer el comportamiento de todos los terminales en un área determinada y los instantes en que estuvieron allí. Además de permitir excluir de entre esos terminales los que sean menos relevantes.
      
      \item Permitir conocer las trayectorias de un terminal objetivo y los instantes en que estuvo allí. Además de permitir identificar posibles terminales que estuvieran con el objetivo en esos instantes.

      \item La aplicación debe tener una interfaz gráfica sobre la que se pueda realizar filtros de búsqueda y mostrar por pantalla los resultados de esos filtros.
    \end{objetive}
  
  
  \section{Organización de la memoria\label{SEC:ORGANIZACION}}
    Este documento está organizado en 6 capítulos (siendo el primero esta introducción): 
    \begin{itemize}
      \item El capítulo \ref{CAP:ESTADOARTE} contiene el \textit{Estado del Arte}, donde se detallarán las herramientas de desarrollo que existen actualmente para implementar la aplicación, destacando las más útiles y analizando sus características.
      
      \item En el capítulo \ref{CAP:AD} de \textit{Análisis y Diseño} se listarán los requisitos funcionales y no funcionales que tiene que cumplir la aplicación. También se decidirán razonadamente las herramientas (detalladas en el capítulo anterior) que se usarán en la implementación junto a los esquemas de la estructura que seguirá esta aplicación. Adicionalmente se comentarán las mayores dificultades encontradas durante el diseño.
      
      \item En el capítulo \ref{CAP:DESARROLLO} de \textit{Desarrollo}, se detallan las versiones de las librerías utilizadas en el momento, y se destacan los aspectos más interesantes que surgen de pasar del diseño a la implementación en código.
      
      \item En el capítulo \ref{CAP:IPR} de \textit{Integración, pruebas y resultados}, se mostrará la interfaz web de la aplicación y se detallará un ejemplo de uso en un posible caso realista e interesante.
      
      \item En el capítulo \ref{CAP:CONCTRAB} de \textit{Conclusiones y trabajo futuro}, se concluye el resultado de este proyecto, y se plantearán posibles mejoras que no ha habido tiempo para implementar.
    \end{itemize}
    
    
