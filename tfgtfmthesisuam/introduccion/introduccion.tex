\chapter{Introducción\label{CAP:INTRODUCCION}}
  \section{Motivación\label{SEC:MOTIVACION}}
    Las fuerzas y cuerpos de seguridad muchas veces disponen únicamente de información sobre el teléfono de una víctima o desaparecido para comenzar la investigación. El objetivo de este trabajo es proveer a un analista de reconocimiento automático de ciertas situaciones a partir de la información provista por las operadoras telefónicas.En particular, se debe ser capaz de:
    i) Determinar el patrón de comportamiento de todos los teléfonos detectados en un área determinada
    ii) Determinar trayectorias y posibles coincidencias de localización de un teléfono específico. 
    El sistema debe estar desarrollado en Python y ofrecer una interfaz Web.
  \section{Objetivos\label{SEC:OBJETIVOS}}
    A continuación listaremos los objetivos que debe cumplir la aplicación para ser satisfactoria.
    \paragraph{Objetivos de la aplicación}
    %TODO
    \begin{objetive}
      \item Permitir definir los formatos de los ficheros de datos a insertar en la Base de Datos.
      \item Permitir insertar los datos en la Base de Datos.
      
      \item Permitir conocer el comportamiento de todos los terminales en un área determinada y los instantes en que estuvieron allí.
      \item Permitir conocer las trayectorias de unos terminales fijados y los instantes cuando estuvieron allí.

      \item La aplicación debe tener una interfaz gráfica sobre la que se pueda interactuar. Consistirá en formularios web para resultar consultas y un mapa interactivo donde se mostrarán dichas consultas.
    \end{objetive}
  
  \section{Organización de la memoria\label{SEC:ORGANIZACION}}
    Este documento está organizado en 6 capítulos, siendo el primero esta introducción. 
    El capítulo \ref{CAP:ESTADOARTE} contiene el Estado del Arte, donde se detallarán las herramientas necesarias que existen actualmente, destacando las más útiles.
    
    En el capítulo \ref{CAP:AD} de Análisis y Diseño...