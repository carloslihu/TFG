\chapter{Desarrollo\label{CAP:DESARROLLO}}
  \section{Versiones de Librerías}
    Al desarrollar esta aplicación, hemos optado por usar las versiones más recientes de cada librería del momento. A continuación listamos las versiones:
    \begin{itemize}
      \item Python 3.7.2
      \item Flask 1.0.2
      \item Flask-SQLAlchemy 2.3.2
      \item PostgreSQL 11.2
      \item Jinja2 2.10
      \item SQLAlchemy 1.3.0 
      \item GeoAlchemy2 0.6.2
      \item PostGIS 2.5
    \end{itemize}
  \section{Implementación}
    \subsection{Base de Datos}
      Para implementar este módulo se ha decidido usar la librería SQLAlchemy\cite{sqlalchemy} de Python. De esta manera se puede abstraer el lenguaje SQL utilizado y realizar todas las operaciones con su \dfn{orm}. 
      Para poder usar la funcionalidad geoespacial asociada a la extensión PostGIS, SQLAlchemy cuenta con su propia extensión de herramienta llamada GeoAlchemy 2\cite{geoalchemy}. Esta herramienta, permite abstraer el uso de PostGIS al igual que SQLAlchemy lo hace con PostgreSQL y extendiendo de forma natural a SQLAlchemy.
    \subsection{Inserción de datos en BD}
      Primero se leen los formatos de los datos de un fichero JSON.
      %mostrar formatos
      A continuación se leen los datos de los ficheros excel con la librería de Panda usando el mapeo designado en el JSON.
      Y finalmente se introducen los datos en la tabla y columnas correspondientes.